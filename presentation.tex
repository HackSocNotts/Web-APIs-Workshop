% 16:9 aspect ratio so it will fill a TV screen
\documentclass[aspectratio=169]{beamer}

\usepackage{hacksoc}

% use pausefn command so you can generate slides with and without pauses
\newcommand{\pausefn}{\pause}
%\newcommand{\pausefn}{}

\title{Sample Presentation}
\subtitle{How to use this \LaTeX{ } template}
\author{Peter Taylor}
\date{\today}

\begin{document}
% Title slide
\maketitleslide

% Section Names have their own slides generated, make sure they are out of a frame
% environment. If you want it numbered, include it in the name of the section
\section{Part IV: Sample Section Title}
\begin{frame}

  \frametitle{Usage}
  Say you wanted to have a footnote\footnote{like this}\footsep\footnote{or this\footnote{Unfortunately, nested footnotes don't work}}, you use \texttt{footnote}, and to separate them use \texttt{footsep}\par

  \epigraph{If you want a nicely formatted quote like this one, use the \texttt{epigraph} environment. However, footnotes don't play nicely with this. Stylistically, I'd put all epigraphs at the end, but you don't have to}{And this is who said it, and when}

  Pretty Much everything else is as expected.
\end{frame}

\begin{frame}
  \frametitle{Some Maths}
  \framesubtitle{Have some lambda calculus}
  The maths is set to use the maths that would be used in an article, as that is simply more readable than what beamer sets it to.
  $$\int Y = \lambda f. (\lambda x. f (x x)) (\lambda x. f (x x))$$

\end{frame}

\end{document}
